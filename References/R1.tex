%[1]
\bibitem{griffiths} D. Griffiths, \textbf{Introduction to Elementary Particle Physics}. Second edition, Wiley-VCH Verlag GmbH \& Co. KGaA, Weinheim. ISBN 978-3-527-40601-2. 2008.

%[2]
\bibitem{nobel_leptons} K. V. L. Sarma, \textbf{Nobel Leptons}. arXiv:hep-ph/9512420. 1995.

%[3]
\bibitem{cowan_reines} C. L. Cowan, F. Reines, F. B. Harrison, H. W. Kruse, and A. D. McGuire. \textbf{Detection of the Free Neutrino: A Confirmation}. Science. 1956.

%[4]
\bibitem{Konopinski_Mahmoud} E. J. Konopinski, H. M. Mahmoud, \textbf{The Universal Fermi Interaction}. Phys. Rev. \textbf{92}. 1953.

%[5]
\bibitem{two_neutrinos} G. Danby, J-M. Gaillard, K. Goulianos, L. M . Lederman, N. Mistry, M. Schwartz, J. Steinberger, \textbf{Observation of high-energy neutrino reactions and the existence of two kinds of neutrinos.} Phys. Rev. Lett. \textbf{9}, 1, 36. 1962.

%[6]
\bibitem{the_story_of_the_neutrino} G. Rajasekaran, \textbf{The Story of the Neutrino}. arXiv:1606.08715. 2016.

%[7]
\bibitem{pontecorvo_1967} B. Pontecorvo, J. Exptl., \textbf{Neutrino Experiments and The Problem of Conservation of Leptonic Charge}. Theoret. Phys. \textbf{53}, 1717 (1967), Sov. Phys. JETP \textbf{26}, 984. 1968.

%[8]
\bibitem{neutrino_oscillations_brief_history_and_present_status} S. M. Bilenky, \textbf{Neutrino oscillations: brief history and present status}. arXiv:1408.2864. 2014.

%[9]
\bibitem{tau_neutrino_discovery}K. Kodama \textit{et al.}, \textbf{Observation of tau neutrino interactions}.Phys. Lett. B \textbf{504}, 218-224. 2001. 

%[10]
\bibitem{MNS} Z. Maki \textbf{et al.}, \textbf{Remarks on the Unified Model of Elementary Particles}. Prog. Theor. Phys. \textbf{28}, 870-880. 1962.

%[11]
\bibitem{PMNS} B. Pontecorvo, \textbf{Inverse beta processes and nonconservation of lepton charge}. Soviet Phys. JETP. \textbf{7}, 172. 1958.

%[12]
\bibitem{oscillation_math} C. Giunti, C. K. Kim, \textbf{Fundamentals of Neutrino Physics and Astrophysics}. Oxford University Press. ISBN 978-0-19-850871-7. 2007.

%[13]
\bibitem{Lauren_thesis} L. Yates, \textbf{Using the MicroBooNE Liquid Argon Detector
to Search for Electron Neutrino Interactions and Understand the MiniBooNE Anomaly}. FERMILAB-THESIS-2022-02. 2022.

%[14]
\bibitem{first_kamioka_measure} Y. Fukuda \textit{et al.}, \textbf{Evidence for oscillation of atmospheric neutrinos}. Phys. Rev. Lett. \textbf{81}, 1562. 1998.  

%[15]
\bibitem{superk_picture} Workers doing PMTs checking during Super-Kamiokande's construction. Digital image. Super-Kamiokande collaboration, n.p. Web 17 May, 2010. 16 Jan 2017. \href{http://www-sk.icrr.u-tokyo.ac.jp/sk/gallery/index-e.html}{http://www-sk.icrr.u-tokyo.ac.jp/sk/gallery/index-e.html}

%[16]
\bibitem{nu_scatter_zeller} J. A. Formaggio and G. P. Zeller, \textbf{From eV to EeV: Neutrino Cross Sections across Energy Scales}, Rev. Mod. Phys. 84, 1307–1341 (2012), arXiv:1305.7513 [hep-ex]

%[17]
\bibitem{delta_2_1} A. Gando \textit{et al.}, \textbf{Neutrino-nucleus quasi-elastic and 2p2h interactions up to 10 GeV}. Phys. Rev. D \textbf{88}, 033001. 2013.

%[18]
\bibitem{delta_3_2} P. Adamson \textit{et al.}, \textbf{Combined Analysis of $\nu_\mu$ Disappearance and $\nu_\mu \rightarrow \nu_e$ Appearance in MINOS Using Accelerator and Atmospheric Neutrinos}. Phys. Rev. Lett. \textbf{112}, 191801 (2014); and update by A. Sousa for MINOS at XXVI International
Conference on Neutrino Physics and Astrophysics (Neutrino 2014), proceedings at arXiv:1502.07715. 2015.

%[19]
\bibitem{prospects_patterson} R. B. Patterson, \textbf{Prospects for measurement of the neutrino mass hierarchy}. arXiv:1506.07917v3. 2016.

%[20]
\bibitem{NOVA} P. Adamson \textit{et al.}, \textbf{Measurement of the neutrino mixing angle $\theta_{23}$ in NOvA}. arXiv:1701.05891. 2017.

%[21]
\bibitem{MINOS} P. Adamson \textit{et al.}, \textbf{Combined Analysis of $\nu_\mu$ Disappearance and $\nu_\mu$ $\rightarrow $ $\nu_e$ Appearance in MINOS Using Accelerator and Atmospheric Neutrinos}. Phys. Rev. Lett. \textbf{112}, 191801. 2014.

%[22]
\bibitem{T2K} K. Abe \textit{et al.},\textbf{Measurements of neutrino oscillation in appearance and disappearance channels by the T2K experiment with $6.6 \times 10^{20} $ protons on target}. Phys. Rev. D \textbf{91}, 072010. 2015.

%[23]
\bibitem{lsnd} A. Aguilar-Arevalo et al. (LSND), \textbf{Evidence for neutrino oscillations from the observation of $\nu_e$ appearance in a $\nu_{\mu}$ beam}, Phys. Rev. D 64, 112007 (2001), arXiv:hep- ex/0104049.

%[24]
\bibitem{miniboone} M. Sorel \textit{et al.}, \textbf{MiniBooNE: first results on the muon-to-electron neutrino oscillation search}, J. Phys.: Conf. Ser. 110 082020, 2008.

%[25]
\bibitem{microboone_lee} P. Abratenko \textit{et al.}, \textbf{Search for an Excess of Electron Neutrino Interactions in MicroBooNE Using Multiple Final State Topologies}. arXiv:2110.14054. 2022.

%[26]
\bibitem{Nygren} D. R. Nygren, \textbf{The Time Projection Chamber: A New 4 pi Detector for Charged Particles}. eConf, vol. C740805, p. 58, 1974.

%[27]
\bibitem{Acciarri_presentation} R. Acciarri (LArIAT Collaboration), \textbf{LArIAT:
Liquid Argon In a Testbeam} Internal report. DocDB-1975

%[28]
\bibitem{Rubia_ANewConcept} C. Rubbia, \textbf{The Liquid Argon Time Projection Chamber: A New Concept for Neutrino Detectors}. 1977.

%[29]
\bibitem{ICARUS_proposal} S. Amerio \textit{et al.}, (ICARUS Collaboration), \textbf{Design, construction and tests of the ICARUS T600 detector}. Nucl. Instr. and Meth. in Phys. Res. \textbf{A 527}, 329. 2004.

%[30]
\bibitem{microboone_proposal} H. Chen \textit{et al.}, \textbf{A Proposal for a New Experiment Using the Booster and NuMI Neutrino Beamlines: MicroBooNE}. 2007.

%[31]
\bibitem{SBND} \textbf{Short-Baseline Near Detector (SBND)}. n.d. Retrieved from \href{http://sbn-nd.fnal.gov}{http://sbn-nd.fnal.gov}

%[32]
\bibitem{dune_snowmass_22} A. Abed Abud \textit{et al.}, \textbf{Snowmass Neutrino Frontier: DUNE Physics Summary: Executive Summary of DUNE Physics Program}. arXiv:2203.06100. 2022

%[33]
\bibitem{SBN} P. Machado, O. Palamara, and D. Schmitz, \textbf{The Short-Baseline Neutrino Program at Fermilab}. arXiv:1903.04608. 2019. 

%[34]
\bibitem{dune_SAND} Matteo Vicenzi, for the DUNE collaboration, \textbf{SAND - System for on-Axis Neutrino Detection - in the DUNE Near Detector complex}. The 22nd International Workshop on Neutrinos from Accelerators (NuFact2021). Italy, 2021.
