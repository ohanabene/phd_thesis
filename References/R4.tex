%[1]
\bibitem{Kamiokande-II-PRL} S. Hirata, et al., Kamiokande-II Collaboration. \textbf{Observation of a neutrino burst from the supernova SN1987A}. Phys. Rev. Lett. \textbf{58}, 1490–1493. 1987.

%[2]
\bibitem{Kamiokande-II-PRD} S. Hirata, et al., Kamiokande-II Collaboration. \textbf{Observation in the Kamiokande-II detector of the neutrino burst from supernova SN1987A}. Phys. Rev. D \textbf{38}, 448–458. 1988.

%[3]
\bibitem{IMB} M. Bionta, et al., IMB Collaboration, \textbf{Observation of a neutrino burst in coincidence with supernova 1987A in the Large Magellanic Cloud}. Phys. Rev. Lett. \textbf{58}, 1494–1496. 1987. 

%[4]
\bibitem{Baksan} Alexeyev et al., \textbf{Detection of the neutrino signal from SN 1987A in the LMC using the INR Baksan underground scintillation telescope}. Phys. Lett. B \textbf{205}, 209–214. 1988.

%[5]
\bibitem{Gardiner_thesis} S. Gardiner. \textbf{Nuclear Effects in Neutrino Detection}. PhD thesis,  U. C. Davis, 2018.

%[6]
\bibitem{kate_scholberg} K. Scholberg. \textbf{Supernova signatures of neutrino mass ordering}. Journal of Physics G: Nuclear and Particle Physics 45, 014002. 2017.

%[7]
\bibitem{dune_supernova_model} L. Hudepohl et al. \textbf{Neutrino signal of electron-capture supernovae from core collapse to cooling}. Physical Review Letters 104, 251101. 2010.

%[8]
\bibitem{Friedland} A. Friedland. \textbf{SN signal modeling: collective effects} Talk presented at the May 2018 DUNE Collaboration Meeting, Batavia, Illinois. 2018. 

%[9]
\bibitem{Yang2012} Z. Yang \textit{et al.}. \textbf{Simulation study of neutrino nucleus cross section measurement in a segmented detector at a spallation neutron source}. Chinese Physics C. \textbf{36}. 538-543. 2012. 


%[10]
\bibitem{g4} S. Agostinelli \textit{et al.}, (Geant4 Collaboration). \textbf{Geant4—a simulation toolkit}. Nucl. Instrum. Meth. A 506, 250-303. 2003.

\bibitem{minerva} L. Aliaga \textit{et al.}, (MINER$\nu$A Collaboration). \textbf{Design, Calibration, and Performance of the MINERvA Detector}.  arXiv:1305.5199. 2013. 

\bibitem{minos} P. Adamson, \textit{et al.}, (MINOS Collaboration), \textbf{MINOS Technical Design Report}. NuMI-NOTE-GEN-0337. 1998.

%[11]
\bibitem{fluka} A. Ferrari \textit{et al.}, (FLUKA Collaboration). \textbf{Fluka:
a multi-particle transport code}. 2021. \href{http://www.fluka.org/content/manuals/FM.pdf
}{http://www.fluka.org/content/manuals/FM.pdf}

%[12]
\bibitem{ppfx} L. Aliaga Soplin. \textbf{Neutrino Flux Prediction for the NuMI Beamline} Ph.D. thesis, William-Mary Coll., 2016.

\bibitem{Walecka-Donnelly} J. S. O’Connell, T. W. Donnelly, and J. D. Walecka. \textbf{Semileptonic weak interactions with C12}. Physical Review C 6, 719-733. 1972.

\bibitem{noise_filter} R. Acciarri \textit{et al.} (MicroBooNE Collaboration). \textbf{Noise Characterization and Filtering in the MicroBooNE Liquid Argon TPC}. JINST 12, P08003. arXiv:1705.07341. 2017. 

\bibitem{avinay_thesis} A. Bhat. \textbf{MeV Scale Physics in MicroBooNE}. PhD thesis, Syracuse U., 2021.

\bibitem{pandora} R. Acciarri \textit{et al.} \textbf{The Pandora multi-algorithm approach to automated pattern recognition of cosmic-ray muon and neutrino events in the MicroBooNE detector}. Eur. Phys. J. C, 78. 2018.

\bibitem{signal_proc} C. Adams \textit{et al.}, (MicroBooNE Collaboration), \textbf{Ionization Electron Signal Processing in Single Phase LArTPCs. Part II. Data/Simulation Comparison and Performance in MicroBooNE}, JINST 13, P07007. 2018.

\bibitem{deexcitation-model} W. Hauser and H. Feshbach. \textbf{The inelastic scattering of neutrons}. Physical Review 87, 366. 1952.
 
\bibitem{will_CM_Aug} W. Foreman. \textbf{MeV-scale 'Blip' Reconstruction in MicroBooNE \&
Measuring the Ambient Radon Rate}. MicroBooNE Collaboration Meeting. Fermilab, August 2022. Internal document DocDB 38470. 2022. 

\bibitem{lariat_calorimetry_lar} W. Foreman, \textit{et al.}, (LArIAT Collaboration). \textbf{Calorimetry for low-energy electrons using charge and light in liquid argon}. arXiv:1909.07920. 2020.

\bibitem{BPULE} Y. Zhu. \textbf{Upper limit for Poisson variable incorporating systematic uncertainties by Bayesian approach}.

\bibitem{ESTAR}  M.J. Berger, \textit{et al.} \textbf{ESTAR, PSTAR, and ASTAR: Computer Programs for Calculating Stopping-Power and Range Tables for Electrons, Protons, and Helium Ions (version 1.2.3).}. Available at \href{http://physics.nist.gov/Star}{http://physics.nist.gov/Star}. 2005.

\bibitem{argoneut_mev} R. Acciarri, \textit{et al.}, (ArgoNeuT Collaboration). \textbf{Demonstration of MeV-Scale Physics in
Liquid Argon Time Projection Chambers Using ArgoNeuT.} FERMILAB-PUB-18-559-ND. arXiv:1810.06502. 2018.

\bibitem{microboone_mev} R. Acciarri \textit{et al.}, (MicroBooNE Collaboration). \textbf{MeV-scale Physics in MicroBooNE}. MICROBOONE-NOTE 1076-PUB. 