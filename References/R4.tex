%[1]
\bibitem{Kamiokande-II-PRL} S. Hirata, et al., Kamiokande-II Collaboration. \textbf{Observation of a neutrino burst from the supernova SN1987A}. Phys. Rev. Lett. \textbf{58}, 1490–1493. 1987.

%[2]
\bibitem{Kamiokande-II-PRD} S. Hirata, et al., Kamiokande-II Collaboration. \textbf{Observation in the Kamiokande-II detector of the neutrino burst from supernova SN1987A}. Phys. Rev. D \textbf{38}, 448–458. 1988.

%[3]
\bibitem{IMB} M. Bionta, et al., IMB Collaboration, \textbf{Observation of a neutrino burst in coincidence with supernova 1987A in the Large Magellanic Cloud}. Phys. Rev. Lett. \textbf{58}, 1494–1496. 1987. 

%[4]
\bibitem{Baksan} Alexeyev et al., \textbf{Detection of the neutrino signal from SN 1987A in the LMC using the INR Baksan underground scintillation telescope}. Phys. Lett. B \textbf{205}, 209–214. 1988.

%[5]
\bibitem{Gardiner_thesis} S. Gardiner. \textbf{Nuclear Effects in Neutrino Detection}. PhD thesis,  U. C. Davis, 2018.

%[6]
\bibitem{kate_scholberg} K. Scholberg. \textbf{Supernova signatures of neutrino mass ordering}. Journal of Physics G: Nuclear and Particle Physics 45, 014002. 2017.

%[7]
\bibitem{dune_supernova_model} L. Hudepohl et al. \textbf{Neutrino signal of electron-capture supernovae from core collapse to cooling}. Physical Review Letters 104, 251101. 2010.

%[8]
\bibitem{Friedland} A. Friedland. \textbf{SN signal modeling: collective effects} Talk presented at the May 2018 DUNE Collaboration Meeting, Batavia, Illinois. 2018. 

%[9]
\bibitem{Yang2012} Z. Yang \textit{et al.}. \textbf{Simulation study of neutrino nucleus cross section measurement in a segmented detector at a spallation neutron source}. Chinese Physics C. \textbf{36}. 538-543. 2012. 


%[10]
