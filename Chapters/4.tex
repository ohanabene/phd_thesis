The only supernova neutrinos detected to date was the SN1987A, from a supernova that happened at the Large Magellanic Cloud. A small amount of electron neutrinos from the event were observed at Kamiokande-II ($12 \nu_e$s), IMB ($8 \nu_e$s), and Baksan ($5 \nu_e$s). Despite the small dataset, a rich amount of analysis was published using the data, \cite{Kamiokande-II-PRL, Kamiokande-II-PRD, IMB, Baksan}. 
 On the occasion of a core-collapse supernova in our galaxy, measuring the flux of neutrinos and antineutrinos from all flavors will significantly enlighten the fields of astrophysics and neutrino physics. Therefore, one of DUNE's goals is to be prepared for such an event, as it is the most potent detector to measure the electron neutrinos signal from a supernova, \cite{dune_SAND}.
 In this chapter, I will explain the basics of supernovas and the neutrinos produced at them, why DUNE is particularly suitable to measure those neutrinos, and how using muos that decay at rest ($\mu$DARs) at the NuMI beam and interact in the MicroBooNE detector can help us improve neutrino-nucleus theoretical models and MeV-scale detection techniques in LArTPCs, which will be essential to accomplish the supernova detection in DUNE. 

\section{Core-Collapse Supernova Neutrino Detection in DUNE}
\subsubsection{Supernovas}
Supernova is the name given to the explosive chain of events that happens during the death of some stars. Mainly, core-collapse supernovas produce a massive number of neutrinos and antineutrinos whose spectrum, if measured, can inform the characteristics of the explosion. 
At the end of the life of a star that has $8-40$ solar masses, its body is made of concentric shells that were formed at each of its previous burning phases, them being: hydrogen, helium, carbon, neon, oxygen, silicon, enveloping a core of nickel and iron. During its life, the stability of the star is a balance between the thermal pressure produced in the fusion process, which pulls everything in the outer direction, and its gravity pulling inward. When the mass of the iron core reaches $1.4$ solar masses, the pressure is not enough to balance the gravity, and the core starts to be compressed by the outer layers. Temperature and density in the core rise and the reaction

\begin{equation}
    e^{-} + p \longrightarrow n + \nu_e
\end{equation}

starts to happen more frequently, causing the pressure from electrons to lower even more and the star to collapse more rapidly. At the latest stages of the collapse, the density is such that even neutrinos get trapped in the core. Eventually, the pressure created by the nucleons degeneracy interrupts the collapse. The sudden stop of the collapse in the core causes a shock wave through the outer layers, and the star explodes. At the early stage of the shock wave, the density lowers, and the neutrinos manage to escape, creating an intense few-millisecond pulse called neutronization burst. 
After the neutronization burst, outer layers of the star still falling innard manage to stall the shock wave briefly. This stage is called the accretion phase. We still don't understand why the shock wave regains energy enough to proceed. One hypothesis is that neutrino reactions produce sufficient heat for the thermal pressure to give continuity to the explosion. 
Next, in the cooling phase, the star's energy is given away in many different processes that create neutrinos and antineutrinos of various flavors. A few of them are:

\begin{align}
    & \text{{\color{gray} electron pair annihilation:}}
    && e^{-} + e^{+} \longrightarrow \nu + \bar{\nu}, \\
    & \text{{\color{gray}electron-electron neutrino bremsstrahlung:}}
    && e^{\pm} + e^{\pm} \longrightarrow e^{\pm} + e^{\pm} + \nu + \bar{\nu} \\
    & \text{{\color{gray}electron-nucleon neutrino bremsstrahlung:}}
    && e^{\pm} + N \longrightarrow e^{\pm} + N + \nu + \bar{\nu} \\
    & \text{{\color{gray}photoannihilation:}}
    && \gamma + e^{\pm} \longrightarrow e^{\pm} + N + \nu + \bar{\nu} \\
    & \text{{\color{gray}nuclear de-excitation by neutrino pair production:}} 
    && ^{Z}_{A}X^{*} \longrightarrow ^{Z}_{A}X + \nu + \bar{\nu}
\end{align}

Given the wide variety of processes, measuring the flux of neutrinos and antineutrinos from all flavors is fundamental to help us better understand aspects of the supernova and what role the processes during the explosion play in its development, \cite{Gardiner_thesis}. 

\subsubsection{Supernova detection in DUNE}

We have in operation or planned for the future a number of large detectors that would be able to detect signals from a core-collapse supernova that happens in our galaxy. The active material of those detectors is either water Cherenkov or hydrocarbon scintillator, making them mainly sensitive to electron antineutrinos via inverse beta decay (the reaction in equation \ref{inverse_beta_decay_eq}). 
Therefore, relying solely on those detectors would preclude the measurement of the supernova electron neutrino flux, which is the most significant component produced during the supernova's neutronization phase. 

Luckily, DUNE's active material is liquid argon, which is mainly sensitive to eletron neutrinos, via charged current scattering in $^{40}$Ar, as in:

\begin{equation}
    \nu_e + ^{40}Ar \longrightarrow ^{40}K + e^{-}
\end{equation}
 
An example of a contribution to neutrino physics that a supernova neutrino detection could make in DUNE is to shine light into the mass hierarchy puzzle. Figure \ref{dune_supernova} shows how the time distribution of the incoming flux of the neutrinos of a $10$ kpc supernova in DUNE could depend on the mass hierarchy. In the case of inverted hierarchy, we should see an early excess of events from the neutronization phase, whether, in the case of normal hierarchy, we should see the absence of such events. The results from figure \ref{dune_supernova} are made assuming a model describer in reference \cite{dune_supernova_model}, but those results are assumed to be fairly model-independent \cite{kate_scholberg}. 

\begin{figure}[h!]
    \centering
    \includegraphics[width=150mm]{Figures/dune_supernova.jpeg}
    \caption[Predicted event time distribution for the detection of supernova neutrinos in a DUNE-like $40$ kt LArTPC]{{\textbf{Predicted event time distribution for the detection of supernova neutrinos in a DUNE-like 40 kt LArTPC}}\\ Predicted event time distribution for the detection of a $10$ kpc supernova neutrinos in a DUNE-like 40 kt LArTPC for each of the supernova stages. The blue curve assumes no oscillations, the green and the red curves assumes oscillations for inverted and normal mass hierarchy, respectively \cite{kate_scholberg}.}
    \label{dune_supernova}
\end{figure}

Furthermore, being able to measure the flux of both neutrinos and antineutrinos is fundamental to further understanding exotic supernova processes \cite{Friedland}. 

\section{$\mu$ Decay At Rest-LAr Physics}

As promising as the DUNE supernova neutrino program is, we currently lack an understand of the MeV-scale neutrinos and LAr nucleus interactions, as well as LArTPC reconstruction capabilities and event identification at this energy range. 

As mentioned before, MicroBooNE is exposed to the BNB and NuMI beamlines. When on the FHC, both beamlines produce a copious amount of antimuons that will, when decay to rest, produce a muon antineutrino, an electron neutrino, and a positron. On figures \ref{numi_nue_flux, numi_nu_flux} you can see the NuMI flux prediction of $\nu_e$s in MicroBoone by parents and the NuMI flux prediction in MicroBooNE of all neutrinos, respectively. 

\begin{figure}[h!]
    \centering
    \includegraphics[width=180mm]{Figures/numi_nue_flux.jpeg}
    \caption[NumI $\nu_e$ flux in MicroBooNE]{{\textbf{NumI $\nu_e$ flux in MicroBooNE}}\\ On the right, the FHC $\nu_e$ neutrino flux at MicroBooNE from NuMI broken down by parent. Althought there is a  60 MeV threshold included in the percentages, the absolute value clearly shows the dominance of muon decays. On the left, the FHC neutrino flux broken down by parent for each neutrino flavour. The percentages shown do not include an integration threshold on the angle. Decays from pions and muons dominate the majority of the flux across all angles \cite{krish_phd}.}
    \label{numi_nue_flux}
\end{figure}

\begin{figure}[h!]
    \centering
    \includegraphics[width=135mm]{Figures/numi_nu_flux.jpeg}
    \caption[NumI $\nu$ flux in MicroBooNE]{{\textbf{NumI $\nu$ flux in MicroBooNE}}\\ The flux prediction for all neutrino flavours in the FHC mode in neutrino angle. No integration thresh- old is applied in the percentages and the electron neutrino flux percentage is boosted by muon decay at rest. The large angular spread in the flux is due to the positioning of MicroBooNE with respect to the NuMI beamline. The flux peaks at the target location and tails off further into the beamline. From angles above $20$ deg, (midway into the decay pipe) the flux remains flat up to the absorber where there is a large peak in the flux spectrum $\approx 120$ deg. After this, the flux is attenuated rapidly going into the muon-shield $ \ge 120$ deg.\cite{krish_phd}.}
    \label{numi_nu_flux}
\end{figure}

Due to the nature of the decay, the electron neutrino will have a kinectic energy $0 \leqslant KE_{\nu_e} \leqslant 50$ MeV, which is the same energy range of the supernova electron neutrinos we aim to detect in DUNE!



decay math 



\section{Simulation of $\mu$DAR Events in MicroBooNE}

To properly evaluate the $mu$DAR events in MicroBooNE, we count with a chain of different simulation softwares to mimic the particle production and interactions at each step of the way. The chain is as it follows: First, we use the NuMI beamline simulation to to simulate the flux of neutrinos from the NuMI in MicroBooNE, this flux prediction is then handed to GENIE, a neutrino event generator, that propagates the neutrino into MicroBooNE. Then, another neutrino generator software, MARLEY, simulates the $mu$DAR $\nu_e$-LAr interaction, returnes the output to GENIE, that passes the result down to GEANT4, to do the detector simulation. I will go over each of those steps and the reasons why they are organized like this int his Chapter.

\subsection{NuMI Beamline Simulation}

GET IT FROM KRISH'S THESIS

\subsection{GENIE and MARLEY}

WHAT GENIE DOES AND HOW IT FAILS, AND HOW MARLEY SAVES THE DAY 

\section{Reconstruction of $\mu$DAR Events in MicroBooNE}

\section{Monte Carlo-Data Agreement}

energy plots
position plots







\section{Background Assessment}
\section{Data Analysis}
\section{Conclusion}